\chapter{Introduction}

\section{Objective}

The primary objective of this experiment is to explore the torsional oscillations of a cylindrical cup suspended on an elastic wire. The specific goals include the determination of the moment of inertia ($I$) of the cylindrical cup using the disk method, the calculation of the torsion modulus ($J$) of the string, and the experimental verification of the conservation of energy.

\section{Context}

Torsional oscillations are a result of a body suspended on an elastic wire undergoing oscillations due to the moment of elastic forces. The relationship between the period ($T$), moment of inertia ($I$), and torsion modulus ($J$) is expressed by this equation: 

\[ T = 2\pi \sqrt{\frac{I}{J}}\]

\noindent According to the conservation law, the kinetic energy of the rotational motion of the pendulum is should be converted into potential energy:

\[ \frac{1}{2}I\omega^2 = \frac{1}{2}J\phi_{\max}^2\]
