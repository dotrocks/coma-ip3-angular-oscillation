\chapter{Experimental Setup}

\section{Materials}

\begin{itemize}
    \item{\textbf{Utensil:}} A cylindrical cup with the following dimensions:
    \begin{itemize}
        \item \textbf{Mass} ($m$): $118g$
        \item \textbf{Inner Radius} ($r_1$): $18cm$
        \item \textbf{Outer Radius} ($r_2$): $20cm$
        \item \textbf{Height} ($h$): $9cm$
    \end{itemize}
    \item{\textbf{Suspension:}} An elastic string attached to the top of the cup.
\end{itemize}

\section{Procedure}

\begin{itemize}
    \item{\textbf{Setup:}} Assemble the angular pendulum setup with the cylindrical cup attached to an elastic lightweight suspension.
    \item{\textbf{Measurements:}} Record the relevant dimensions of the cylindrical cup, including the inner and outer radii.
    \item{\textbf{Angular Displacement:}} Release the cylindrical cup to measure angular displacement ($\phi_{\max}$) and record the corresponding angular velocity ($\omega$).
    \item{\textbf{Data Collection:}} Document all experimental data for subsequent analysis.
\end{itemize}

\newpage
\thispagestyle{plain}

\section{Data}

The data below is obtained from the experimental setup described above. The data is used to calculate the moment of inertia ($I$) of the cylindrical cup and torsion modulus ($J$) of the string. Here is the experiment video: \url{https://youtu.be/hGKMeG9jghA}

\bigbreak{}
\bigbreak{}

\begin{center}
    \begin{tabular}{ |l||l|  }
        \hline
    
        Data & Value \\
    
        \hline
    
        Mass ($m$) & $118~g$ \\
        Frequency ($f$) & $\frac{1}{3}~Hz$ \\
        Maximum angle of twist ($\phi_{\max}$) & $360^\circ$ \\
        Angular velocity after impact ($\omega$) & $19~rad/s$ \\
        \hline
    \end{tabular}    
\end{center}

